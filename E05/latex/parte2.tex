\subsection{Amplificatore differenziale}

\begin{wrapfigure}[14]{l}{0.5\textwidth}
  \begin{center}
    \includegraphics[width=0.280\textwidth]{../E05/latex/c_teo_diff_amp.pdf}
  \end{center}
  \caption{...}
  \label{cir5:diff_amp_teo}
\end{wrapfigure}

In questa parte dell'esperienza cercheremo di capire e analizzare un amplificatore differenziale (\ref{cir5:diff_amp}). Oltre al fatto che scegliendo i valori di resistenza possiamo decidere il guadagno, la proprietà più importante è sicuramente il vantaggio che si ha rispetto al rumore in modo comune. Per capirne il motivo analizziamo il circuito riportato in figura (\ref{cir5:diff_amp_teo}).

Sfruttiamo il principio di sovrapposizione per valutare la tensione in uscita in funzione delle due di ingresso. Ricordiamo che la sovrapposizione può essere applicata per sistemi lineari. Ciò è molto utile in quanto ci permette di separare l'analisi circuitale e ricondurci a casi più semplici.

Chiamiamo $V_-$ la tensione all'ingresso invertente e $V_+$ quella all'ingresso non invertente. Poniamo $V_1=0$. La tensione le punto $V_+$ sarà ovviamente data dal partitore con $R$ ed $mR$. Assumendo l'op-amp ideale, abbiamo $V_+=V_-$. Possiamo dunque scrivere:

\begin{equation}
\begin{cases} V_+=V_-  \\ V_+=V_2\frac{mR}{R+mR} \\ V_-= V_{out2}\frac{R}{R+mR}\end{cases}  \Rightarrow V_{out2}=mV_2
\label{eq 5: vout2}
\end{equation}

\begin{wrapfigure}[13]{r}{0.5\textwidth}
  \begin{center}
    \includegraphics[width=0.280\textwidth]{../E05/latex/c_diff_amp.pdf}
  \end{center}
  \caption{...}
  \label{cir5:diff_amp}
\end{wrapfigure}

Analogamente, poniamo $V_2=0$. Otteniamo le seguenti equazioni del circuito:

\begin{equation}
\begin{cases} V_+=V_-=0  \\ \frac{V_1}{R}+\frac{V_{out2}}{mR}=0 \end{cases}  \Rightarrow V_{out1}=-mV_1
\label{eq 5: vout1}
\end{equation}

Sommando ora (\ref{eq 5: vout2}) e (\ref{eq 5: vout2}) otteniamo trivialmente $V_{out}=m(V_2-V_1)$.

Possiamo ora passare ad analizzare il circuito da cui siamo partiti, ovvero quello affetto da noise riportato in figura (\ref{cir5:diff_amp}). Possiamo scrivere, senza perdere di generalità,
$$
\begin{cases} V_1=V_{in}+V_{noise} \\ V_2=V_{noise} \end{cases}  \Rightarrow V_{out}=m(V_{noise}-(V_{in}+V_{noise}))=-V_{in}
$$

L'amplificatore differenziale ci permette dunque di eliminare in modo efficace il rumore di modo comune. 

Come vediamo, la resistenza collegata tra comune e $V_+$ è in realtà un trimmer. Ciò è necessario in quanto, non essendo esattamente uguali le resistenze ed essendo reale l'op-amp, prima di utilizzare il circuito è necessario bilanciare il circuito in modo da ottenere un segnale il più piccolo possibile quando i segnali in ingresso sono uguali. 

Il laboratorio abbiamo utilizzato come sorgente di noise il generatore di forme d'onda e come $V_{in}$ il generatore di tensione costante. Abbiamo utilizzato delle R da \SI{10}{\kilo\ohm} e $mR=2R$. La resistenza variabile è stata costruita mettendo in serie una da \SI{10}{\kilo\ohm} con un trimmer sempre da \SI{10}{\kilo\ohm}. Per controllare il bilanciamento abbiamo connesso entrambi gli ingressi al generatore di forme d'onda ed è stato utilizzato un segnale sinusoidale di $20Vpp$. Il segnale in uscita è stato dunque riportato sull'oscilloscopio. Abbiamo notato che cambiando il valore della resistenza con il trimmer il segnale in uscita cambiava. Il miglior bilanciamento che siamo riusciti a raggiungere a portato la tensione picco-picco del rumore a \SI{0.5}{\milli\volt}. Abbiamo dunque riattaccato il generatore di tensione costante e alimentato il circuito con $-2Vpp$. Il rumore è sempre stato simulato con una sinusoidale di $20Vpp$. I risultati sono riportati nella seguente figura.

\begin{figure}[ht]
 \centering
   {\includegraphics[width=0.75\textwidth]{../E05/latex/amp_diff.pdf}}
 \caption{Come vediamo dal grafico, il rumore che abbiamo simulato ha un'ampiezza di 20 Volt picco-picco (sinusoide in nero). Nonostante ciò, l'uscita (in verde) non è affatto influenzata e risulta amplificata di 2 volte, come stimato precedentemente dai calcoli teorici. }
 \label{gr5:amp_diff}
\end{figure}

Come vediamo, il rumore è stato completamente eliminato dall'amplificatore differenziale. Abbiamo successivamente provato a sbilanciare il circuito, per vedere l'effetto del rumore sul segnale in uscita. E' stato dunque variata la resistenza con trimmer e abbiamo utilizzato un rumore con dei picchi molto accentuati, così da vedere bene gli effetti sul segnale in uscita. 

\begin{figure}[ht]
 \centering
   {\includegraphics[width=0.75\textwidth]{../E05/latex/sbil_amp_diff.pdf}}
 \caption{Dopo lo sbilanciamento del circuito non è più garantita la soppressione del rumore in modo comune. Come vediamo, il segnale in uscita risulta distorto. E' dunque necessario calibrare bene il circuito prima di effettuare esperimenti nei quali si vuole eliminare un rumore comune ad entrambi i segnali in ingresso.}
 \label{gr5:sbil_amp_diff}
\end{figure}

Come vediamo, il segnale risulta disturbato dal rumore. Inoltre, il guadagno può essere cambiato solo se si cambiano 2 resistenze nel circuito. Una soluzione efficace che inoltre rimedia alla bassa impedenza in ingresso è quella di utilizzare un Amplificatore per Strumentazione.
