\subsection*{Conclusioni}
In questa esperienza abbiamo studiato alcuni circuiti utili a trasformare un segnale senza modificare l'informazione in esso contenuta: in particolare abbiamo esaminato circuiti raddrizzatori che non diminuiscono l'ampiezza del segnale come accade con i circuiti passivi quali i semplici diodi o il ponte raddrizzatore (ponte di Graetz).

Successivamente abbiamo studiato il cosiddetto amplificatore da strumentazione, in un circuito didattico costruito con un $\mu$A741 le cui limitazioni sono date dal guadagno fisso e dalla bassa impedenza in entrata e integrato in un dual inline package (AD622).
Ne abbiamo testato il funzionamento misurando lo sbilanciamento di un ponte di Wheatstone.

Infine abbiamo studiato una problematica di una misura a distanza data dalla lunghezza dei cavi.
Abbiamo risolto il problema con una misura a quattro fili.
