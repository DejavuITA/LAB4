\section{14.10.2014 - Raddrizzatore, Amplificatori e Termoresistenza}

In questa esperienza progetteremo un raddrizzatore di precisione (utilizzabile anche per segnali di bassa tensione). Dopodiché valuteremo la capacità di un amplificatore differenziale e dell'integrato $AD622$ di abbattere i guadagni di modo comune. Infine utilizzeremo una termoresistenza per ricavare la temperatura in alcune situazioni.

\subsection*{Strumenti e materiali}

\begin{itemize} [noitemsep]
\item Oscilloscopio Agilent DSO-X 2002A (bandwidth \SI{70}{\mega\hertz}, sample rate \num{2} GSa/s);
\item Generatore di tensione continua Agilent E3631A (max $\pm \, \SI{25}{\volt}$ o $\pm \, \SI{6}{\volt}$);
\item Generatore di forme d'onta Agilent 33120A con range di frequenza da \SI{100}{\micro\hertz} a \SI{15}{\mega\hertz};
\item Multimetro Agilent 34410A a sei cifre e mezza;
\item Un amplificatore operazionale $\mu$A741;
\item Un integrato AD622;
\item Una termoresistenza PT100;
\item Resistenze e capacità di vari valori;
\item Due trimmer multigiro da \SI{10}{\kilo\ohm};
\item Breadboard e cablaggi vari.
\end{itemize}
