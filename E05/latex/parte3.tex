\subsection{Instrumentation Amplifier}

Il circuito visto al paragrafo precedente presenta dei problemi. Per prima cosa l'impedenza in ingresso vista da un eventuale generatore di segnale è bassa, e ciò rende necessaria una potenza maggiore da parte del generatore per mantenere la tensione ai suoi capi; il secondo luogo, il guadagno non è impostabile (se non cambiando le resistenze ad ogni applicazione: procedura scomoda nelle applicazioni pratiche).

Per ovviare a questi problemi, valutiamo il circuito in Figura ??????. Si nota subito che l'impedenza in ingresso è molto alta, in quanto gli amplificatori operazionali hanno un valore di impedenza molto alta ai loro ingressi (non a caso sono usati in configurazione follower per adattare le impedenze).

Verifichiamo invece che la resistenza posta in mezzo varia effettivamente il guadagno del circuito. Dividiamo il circuito in due parti, calcolandoci prima la tensione fra il punto A e B, per poi la tensione di uscita dell'intero circuito. Per la prima parte, considerando i due operazionali ideali, otteniamo che le tensione agli ingressi invertenti sono $V_{OP_1}^- = V_1$ e $V_{OP_2}^- = V_2$. Dunque ai capi di $R_g$ è presente una differenza di potenziale $V_2-V_1 = \Delta V_{in}$; per l'idealità degli OPAMP, abbiamo inoltre che la corrente che passa per le resistenze $R_1$ ed $R_g$ sono uguali, da cui otteniamo
$$\Delta V_{AB} = \frac{\Delta V_{in}}{R_g} R_g + \frac{\Delta V_{in}}{R_g} R_1 + \frac{\Delta V_{in}}{R_g} R_2$$