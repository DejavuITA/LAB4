\subsection{Instrumentation Amplifier}

Il circuito visto al paragrafo precedente presenta dei problemi. Per prima cosa l'impedenza in ingresso vista da un eventuale generatore di segnale è bassa, e ciò rende necessaria una potenza maggiore da parte del generatore per mantenere la tensione ai suoi capi; il secondo luogo, il guadagno non è impostabile (se non cambiando le resistenze ad ogni applicazione: procedura scomoda nelle applicazioni pratiche).

Per ovviare a questi problemi, valutiamo il circuito in Figura ??????. Si nota subito che l'impedenza in ingresso è molto alta, in quanto gli amplificatori operazionali hanno un valore di impedenza molto alta ai loro ingressi (non a caso sono usati in configurazione follower per adattare le impedenze).

Verifichiamo invece che la resistenza posta in mezzo varia effettivamente il guadagno del circuito. Dividiamo il circuito in due parti, calcolandoci prima la tensione fra il punto A e B\footnote{Si noti che un circuito che abbia come $V_{out}$ la differenza fra questi due punti sarebbe flottante sull'uscita. Infatti, una eventuale resistenza di carico messa fra il punto A e B non avrebbe un riferimento di comune.}, per poi la tensione di uscita dell'intero circuito. Per la prima parte, considerando i due operazionali ideali, otteniamo che le tensione agli ingressi invertenti sono $V_{OP_1}^- = V_1$ e $V_{OP_2}^- = V_2$. Dunque ai capi di $R_g$ è presente una differenza di potenziale $V_2-V_1 = \Delta V_{in}$; per l'idealità degli OPAMP, abbiamo inoltre che la corrente che passa per le resistenze $R_1$ ed $R_g$ sono uguali. Otteniamo dunque, sommando le cadute di potenziale
$$\Delta V_{AB} = \frac{\Delta V_{in}}{R_g} R_g + \frac{\Delta V_{in}}{R_g} R_1 + \frac{\Delta V_{in}}{R_g} R_2$$
cioè
\begin{equation}
\Delta V_{AB} = \Delta V_{in} \left(1+\frac{2R_1}{R_g}\right)
\label{eq5:TEMP_calcoli}
\end{equation}

Valutiamo ora la seconda parte del circuito data da $OP_3$. Per quanto riguarda la tensione all'ingresso non invertente, questa è data da un semplice partitore (non scorre corrente nell'ingresso dell'opamp)
$$V^+=V_B \frac{R_3}{R_3+R_2}$$
All'ingresso non invertente vale invece la legge di Kirkhhoff per i nodi
$$\frac{V_A-V^-}{R_2}+\frac{V_{out}-V^-}{R_3}=0$$
da cui
$$V^-=V_A \frac{R_3}{R_2+R_3} + V_{out} \frac{R_2}{R_2+R_3}$$

Uguagliando $V^-$ e $V^+$ (OPAMP ideale), tenendo conto della (\ref{eq5:TEMP_calcoli}), otteniamo dunque
$$V_{out}=-\Delta V_{in} \left(1+\frac{2R_1}{R_g}\right)\frac{R_3}{R_2}$$
Modificando la resistenza $R_g$ possiamo dunque controllare il guadagno del circuito.

In realtà, nell'esperienza abbiamo utilizzato un circuito integrato (più precisamente l'AD622) che ha come qualità il fatto di avere un'alta precisione sul valore delle resistenze (abbiamo visto nel precedente circuito che uno squilibrio anche minimo fra i valori di resistenze che dovrebbero essere uguali può portare ad un guadagno diverso da quello teorico). Il suo guadagno è dato dall'equazione, fornita dal costruttore
$$G=1+\frac{50.5 \si{\kilo\ohm}}{R_g}$$