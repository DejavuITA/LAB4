\section{11.11.2014 - Porte logiche}

In questa esperienza verificheremo il funzionamento di alune porte logiche TTL e semplici circuiti costruiti con esse.

\subsection*{Strumenti e materiali}

\begin{itemize} [noitemsep]
	\item Oscilloscopio Agilent DSO-X 2002A (bandwidth \SI{70}{\mega\hertz}, sample rate \num{2} GSa/s);
	\item Generatore di tensione continua Agilent E3631A (max $\pm \, \SI{25}{\volt}$ o $\pm \, \SI{6}{\volt}$);
	\item Multimetro Agilent 34410A a sei cifre e mezza;
	\item Un integrato 7400, composto di quattro porte logiche TTL NAND; % '00;
	\item Basetta a LED;		
	\item Resistenze e capacità di vari valori;
	\item Breadboard e cablaggi vari.
\end{itemize}
