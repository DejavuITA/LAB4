\subsection*{Conclusioni}
In questa esperienza abbiamo costruito un circuito che, come l'oscillatore a rilassamento, fornisce un'oscillazione autosostenuta. La cosa interessandte del ponte di Wien è che la forma d'onda in uscita è praticamente sinusoidale. Ciò può essere utile per applicazioni dove si necessità di una forma d'onda di tale tipo. Nella seconda parte dell'esperienza abbiamo compesato una sonda. Tali sonde sono molto utili per il fatto che sono insensibili alla frequenza del segnale. Se andiamo a rivedere l'Esperienza 3, nel grafico \ref{gr3:slew_rate} vediamo come l'onda quadra in ingresso non sia perfetta. Ciò è dovuto ad un effetto di sovracompensazione della sonda (in quel caso un semplice cavo). Dunque, per esperimenti che riguardano segnali ad alta frequenza o che contengono alte frequenze (onda quadra), è preferibile utilizzare sonde compensate.
