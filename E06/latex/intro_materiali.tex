\section{22.10.2014 - Termostatazione}

In questa esperienza costruiremo un termometro elettronico utilizzando una Pt100. Realizzeremo poi un sistema di controllo proporzionale di temperatura utilizzando il termometro da noi costruito. 

\subsection*{Strumenti e materiali}

\begin{itemize} [noitemsep]
\item Oscilloscopio Agilent DSO-X 2002A (bandwidth \SI{70}{\mega\hertz}, sample rate \num{2} GSa/s);
\item Generatore di tensione continua Agilent E3631A (max $\pm \, \SI{25}{\volt}$ o $\pm \, \SI{6}{\volt}$);
\item Generatore di forme d'onta Agilent 33120A con range di frequenza da \SI{100}{\micro\hertz} a \SI{15}{\mega\hertz};
\item Multimetro Agilent 34410A a sei cifre e mezza;
\item Un amplificatore operazionale OP07;
\item Un integrato AD622;
\item Un integrato REF02;
\item Una termoresistenza PT100;
\item Transistor di potenza NPN 2N2222;
\item Resistenze e capacità di vari valori;
\item Due trimmer multigiro da \SI{10}{\kilo\ohm};
\item Breadboard e cablaggi vari.
\end{itemize}
