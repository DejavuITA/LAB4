\section{24.09.2014 - Amplificatori reali}

Scopo di questa esperienza è quello di studiare un amplificatore operazionale $\mu a 741$ reale.
Ne analizzeremo l'offset e le correnti di polarizzazione (\textit{bias currents}) cercando di stimarne un valore, tramite circuiti progettati ad hoc.
Premettiamo che il circuito di alimentazione è lo stesso utilizzato nella precedente esperienza e dunque non ripeteremo le considerazioni e gli schemi circuitali già proposti.
Inoltre ricordiamo che la circuiteria di alimentazione sugli schemi è stata nascosta per facilitarne la comprensione.

\subsection{Strumenti e materiali}

\begin{itemize} [noitemsep]
%\item Oscilloscopio Agilent DSO-X 2002A (bandwidth \SI{70}{\mega\hertz}, sample rate \num{2} GSa/s);
\item Generatore di tensione continua Agilent E3631A (max $\pm \, \SI{25}{\volt}$ o $\pm \, \SI{6}{\volt}$);
%\item Generatore di forme d'onta Agilent 33120A con range di frequenza da \SI{100}{\micro\hertz} a \SI{15}{\mega\hertz};
\item Multimetro Agilent 34410A a sei cifre e mezza;
\item Un amplificatore operazionale $\mu$A741;
\item Resistenze e capacità di vari valori;
\item un trimmer a un giro da \SI{5}{\kilo\ohm} e uno da \SI{10}{\kilo\ohm};
\item un trimmer multigiro da \SI{10}{\kilo\ohm};
\item Breadboard e cablaggi vari.
\end{itemize}
