\subsection{Stima e correzione dell'offset}
\label{par2:offset}

In questa prima parte dell'esperienza tratteremo il problema dell'offset. In un amplificatore ideale sappiamo che quando sia ingresso invertente che ingresso non invertente sono collegati a comune il segnale in uscita è nullo. Ciò è dovuto alla perfetta simmetria interna dell'op-amp. Ovviamente nel mondo reale non è possibile realizzare tale fatto in quanto non si riescono a costruire transistor BJT con specifiche identiche. 

\subsubsection{Configurazione senza retroazione}

Quando colleghiamo entrambi gli ingressi a comune l'op-amp vede all'ingresso una differenza di potenziale (ovviamente, fra gli ingressi non c'è, in quanto collegati entrambi a comune) la quale viene amplificata dal guadagno a maglia aperta: come $V_{out}$ avremo dunque un valore diverso da zero. Nel nostro caso l'op-amp andava in saturazione negativa (\SI{-12.9}{\volt}). Ricordando il funzionamento di un amplificatore operazionale, possiamo dire che il circuito si comporta come se la tensione all'ingresso invertente fosse maggiore di quella all'ingresso non invertente. Inoltre il valore $V_{out}$ è diverso da \SI{-15}{\volt} utilizzati come alimentazione in quanto il valore di tensione massimo $|V_{out}|$ è leggermente inferiore a $|V^-|$. In Figura(\ref{cir:open_loop}) è riportato lo schema del circuito utilizzato.

\begin{figure}[ht]
        \centering
        \begin{subfigure}[b]{0.35\textwidth}
                 \includegraphics[width=0.70\textwidth]{../E02/latex/open_loop.pdf}
                \caption{Circuito a maglia aperta}
                \label{cir:open_loop}
        \end{subfigure}%
    \quad
        \begin{subfigure}[b]{0.35\textwidth}
               \includegraphics[width=0.70\textwidth]{../E02/latex/inv.pdf}
                \caption{Circuito amplificatore}
                \label{cir:inv}
        \end{subfigure}
     
\end{figure}

Con il circuito in Figura (\ref{cir:open_loop}) non possiamo quindi ricavare una stima del valore di offset. Per fare ciò dobbiamo ricorrere a un circuito amplificatore come quello in Figura(\ref{cir:inv}) (invertente o non invertente), che ci permetta di controllare il guadagno.

\subsubsection{Configurazione con retroazione}

Trattiamo per primo il caso \textbf{invertente}. Assumiamo che la tensione nel punto A sia $V_A=V_{off}$ e $V_B=0$. Vale allora che, uguagliando le correnti

$$\frac{V_{off}}{R_1} + \frac{V_{off}-V_{out}}{R_2} = 0$$

da cui si ricava che
$$V_{out}=\left(1+\frac{R_2}{R_1}\right) V_{off}$$

Analogamente si tratta il caso \textbf{non invertente}. Per far ciò assumiamo $V_B=-V_{off}$ e $V_A=0$. L'analisi risulta dunque identica a quella di un amplificatore non invertente, e si ottiene lo stesso risultato del caso invertente. 

%I valori nominali delle componenti circuitali utilizzate sono: $R_1=\SI{120}{\ohm}$ e $R_2=\SI{10/100}{\kilo\ohm}$.

Riportiamo nella seguente tabella i valori di offset calcolati, definendo il guadagno come Gain$=1+R_2/R_1$. 

\begin{center}
\begin{savenotes}
\begin{tabular}{c|c|c|c|c|c|c}
$R_1[\si{\ohm}]$ & $R_2[\si{\kilo\ohm}]$ & Gain &$V_{out} [\si{\milli\volt}]$ & $V_{off}$ [\si{\milli\volt}]\\ 
\hline 
$119.8\pm0.1$ & $9.911\pm0.001$ & $83.73\pm0.07$&  $-103.5 \pm 0.5$ & $-1.23 \pm0.01$\\
\hline
$119.8\pm0.1$ & $99.35\pm0.01$ & $830.3\pm0.7$ &$ -1025 \pm 2$ & $-1.2 \pm0.1$\\

\end{tabular}
\end{savenotes}
\end{center}

\begin{wrapfigure}[17]{l}{0.55\textwidth}
  \begin{center}
    \includegraphics[width=0.280\textwidth]{../E02/latex/trimmer_correction.pdf}
  \end{center}
  \caption{Circuito a guadagno unitario, con trimmer sui piedini 1 e 5 dell'OPAMP per compensare l'offset.}
  \label{cir2:trimmer}
\end{wrapfigure}

Come sappiamo la tensione di offset non è però l'unico problema che incontriamo quando usiamo op-amp reali. Infatti ingresso invertente e non invertente sono collegati alle basi di transistor e, ovviamente, per polarizzarli serve una corrente di base. Gli effetti di tale corrente si sommeranno dunque a quelli dovuti all'offset. Tale argomento sarà comunque trattato approfonditamente nella sezione successiva. Per risolvere il problema dell'offset possiamo servirci di una resistenza variabile ($trimmer$) che posizioneremo tra i piedini 1 e 5 dell'op-amp, collegandola a $V^-$. Regolando tale resistenza andremo a generare una contro tensione che bilancerà l'offset. Il circuito utilizzato per verificare il bilanciamento dell'offset è quello in Figura \ref{cir2:trimmer}.

Durante l'esperienza abbiamo provato ad utilizzare trimmer ad un giro da \SI{10}{\kilo\ohm} ma la sensibilità meccanica era troppo bassa per poter azzerare l'offset (la tensione di uscita infatti passava da $\approx$\SI{-13}{\volt} a $\approx$\SI{+13}{\volt}). Abbiamo dunque utilizzato un trimmer multigiro da \SI{10}{\kilo\ohm}, con il quale abbiamo raggiunto la tensione $V_{out}= (1.3\pm0.2)\si{\volt}$. Ricordando che il guadagno a maglia aperta è di 100-120dB, si nota il buon bilanciamento della tensione di offset.

\begin{wrapfigure}[17]{r}{0.55\textwidth}
  \begin{center}
    \includegraphics[width=0.280\textwidth]{../E02/latex/current_correction.pdf}
  \end{center}
  \caption{Circuito utilizzato per la stima di $V_{off}$ con la correzione sulle correnti data da $R_C$.}
  \label{cir2:current_correction}
\end{wrapfigure}

Come già accennato le correnti di bias, per quanto piccole (\si{\nano\ampere}) giocano comunque un ruolo sull'offset totale. Per minimizzare il loro impatto sul valore della tensione di offset abbiamo inserito fra l'ingresso non invertente e ground una resistenza di compensazione $R_C$ tale da annullare il contributo delle correnti di offset sulla tensione di uscita.

Il contributo alla tensione di uscita data dalle correnti di polarizzazione è

\begin{equation}
V_{out}^{C} = \left( 1+\frac{R_2}{R_1} \right)\left[ \frac{I_{b^-}R_2}{\frac{R_1+R_2}{R_1}} - I_{b^+} R_B\right]
\label{eq2:Vout_currents}
\end{equation}

dalla quale si ricava che, supponendo $I_{off} = |I_{b^+}|-|I_{b^-}| \approx 0$ e ponendo $V_{out}^{C}=0$,

$$R_C=\frac{1}{\frac{1}{R_1} + \frac{1}{R_2}} $$

cioè il parallelo fra $R_1$ ed $R_2$.

Riportiamo nella seguente tabella i nuovi valori in questa nuova configurazione (circuito in Figura \ref{cir2:current_correction}):

\begin{center}
\begin{tabular}{c|c|c|c|c|c|c}
$R_C [\si{\ohm}]$& $R_1[\si{\ohm}]$ & $R_2[\si{\kilo\ohm}]$ & Gain & $V_{out}' [\si{\milli\volt}]$ & $V_{off}' [\si{\milli\volt}]$ & $|V_{off}-V_{off}'|[\si{\milli\volt}]$ \\ 
\hline 
$119.4\pm0.1$ & $119.8\pm0.1$ & $9.911\pm0.001$  & $83.73 \pm 0.07$ & $-105.5 \pm 0.5$ & $-1.26 \pm0.01$ & $0.02\pm0.01$ \\
\hline
$119.4\pm0.1$ & $119.8\pm0.1$ & $99.35\pm0.01$  & $830.3\pm0.7$ &$ -1038 \pm 5$ & $-1.2 \pm 0.1$ & $\approx 0$\\
\end{tabular}
\end{center}

Notiamo che le differenze fra i valori della tensione di offset con e senza $R_C$ sono compatibili con il rumore ambientale di fondo (qualche decina di \si{\micro\volt} di ampiezza).

%Come vediamo, le differenze tra i due valori di $V_{off}$ calcolati con e senza resistenza $R_C$ sembrerebbe compatibile con il contributo di una corrente $I^*$ dell'ordine dei \si{\nano\ampere} che scorre nella resistenza $R_C$: $\Delta V \simeq R_C  I^* \simeq 10\div100 \si{\micro\volt}$. Anche nel caso non invertente $\Delta V$ risulta compatibile con la caduta di potenziale data da una corrente di base di alcuni \si{\nano\ampere}. Bisogna però tener conto anche del rumore ambientale, che raggiunge qualche decina di \si{\micro\volt}, e che dunque rende poco chiaro se la differenza tra le tensioni di offset sia dovuta ad un contributo delle correnti di polarizzazione o unicamente al rumore.

%Rimandiamo al paragrafo \ref{par2:v_off} per ulteriori analisi su queste considerazioni.