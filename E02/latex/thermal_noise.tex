\subsubsection*{Rumore termico}

Abbiamo considerato un possibile effetto del rumore termico sulla nostra resistenza da \SI{10}{\mega\ohm}. Considerando la larghezza di banda del multimetro $\Delta f = \SI{1}{\hertz}$ e siano $k_B = 1.38 10^{-23} \si{\joule\per\kelvin}$ (oppure $k_B = \SI{1.38e-23}{\joule\per\kelvin}$) la costante di Boltzmann, $T$ la temperatura assoluta e $R$ il valore della resistenza presa in considerazione.

\begin{equation}
	V_{eff}^2 = 4 k_B T R \Delta f
\end{equation}

da cui, ovviamente, si ottiene che il possibile rumore termico è 5 ordini di grandezza inferiore della tensione misurata ed è pertanto ininfluente.

\begin{equation}
	V_{eff} = sqrt{4 k_B T R \Delta f} \simeq 4 10^{-7}
\end{equation}