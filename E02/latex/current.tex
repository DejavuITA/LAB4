\section{Correnti di polarizzazione}

In questa parte dell'esperienza abbiamo progettato diversi circuiti per misurare la corrente di polarizzazione per entrambi gli ingressi. Di seguito proponiamo due modalità.

\subsection{Configurazione senza retroazione}

Nel circuito mostrato in figura abbiamo posto la resistenza all'ingresso non invertente (analogamente si può fare con l'ingresso invertente) e, misurando la caduta di potenziale ai capi della stessa con il multimetro, possiamo ottenere il valore di corrente desiderato applicando semplicemente la legge di Ohm

$$V=I_{b^+} R$$

Per far ciò, dato che attendevamo una corrente dell'ordine dei \si{\nano\ampere}, abbiamo utilizzato una resistenza molto grande in modo da poter leggere il valore della tensione su una scala accettabile per il multimetro.

Durante la procedura abbiamo però notato che, a causa di rumori ambientali, il valore di tensione sul multimetro fluttuava sulla prima cifra, rendendo nostra misurazione ovviamente non quantitativa (al massimo poteva stimarci l'ordine di grandezza della corrente). Per ovviare, abbiamo inserito in parallelo alla resistenza un condensatore che caricandosi si portava alla stessa ddp dei capi della resistenza. In questo modo abbiamo potuto ottenere un valore meno fluttuante, che si attestava a $V=(-80 \pm 2)$ \si{\milli\volt}, cioè $I_{b^+}=(7.7 \pm 0.4)$ \si{\nano\ampere}.

Con questo metodo semplice abbiamo potuto ottenere una prima stima del valore della corrente