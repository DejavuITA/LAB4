\section{Correnti di polarizzazione}

In questa parte dell'esperienza abbiamo progettato diversi circuiti per misurare la corrente di polarizzazione per entrambi gli ingressi. Di seguito proponiamo due modalità.

\subsection{Configurazione senza retroazione}

Nel circuito mostrato in figura abbiamo posto la resistenza all'ingresso non invertente (analogamente si può fare con l'ingresso invertente) e, misurando la caduta di potenziale ai capi della stessa con il multimetro, possiamo ottenere il valore di corrente desiderato applicando semplicemente la legge di Ohm

$$V=I_{b^+} R$$

Per far ciò, dato che attendevamo una corrente dell'ordine dei \si{\nano\ampere}, abbiamo utilizzato una resistenza molto grande in modo da poter leggere il valore della tensione su una scala accettabile per il multimetro.

Durante la procedura abbiamo però notato che, a causa di rumori ambientali, il valore di tensione sul multimetro fluttuava sulla prima cifra, rendendo nostra misurazione ovviamente non quantitativa (al massimo poteva stimarci l'ordine di grandezza della corrente). Per ovviare, abbiamo inserito in parallelo alla resistenza un condensatore che caricandosi si portava alla stessa ddp dei capi della resistenza. In questo modo abbiamo potuto ottenere un valore meno fluttuante, che si attestava a $V=(-80 \pm 2)$ \si{\milli\volt}, cioè $I_{b^+}=(7.7 \pm 0.2)$ \si{\nano\ampere}.

Con questo metodo semplice abbiamo potuto ottenere una prima stima del valore della corrente. Di contro bisogno considerare che il rumore non permette di avere una stima qualitativa ed inoltre la resistenza, scaldandosi, modifica il suo valore e potrebbe portare ad un errore sulla misura. Nel paragrafo successivo progetteremo dunque un circuito che, sfruttando l'amplificazione data dall'amplificatore operazionale, minimizzerà questi errori.

\subsection{Configurazione con retroazione negativa}

Sfruttando un modello simile a quello utilizzato per trovare la tensione di offset, abbiamo montato i circuiti come in figura.

\subsubsection{Invertente}

Risolviamo il circuito per trovare la corrente di polarizzazione $I_{b^-}$ in funzione della tensione di uscita. Considerando $V^*$ la tensione al capo di $R_B$ opposto a quello collegato all'OPAMP, vale in quel punto la legge di Kirchhoff sui nodi

$$\frac{V^* - V_{in}}{R_1} + \frac{V^*-V_{out}}{R_2} + \frac{V^*-V_T}{R_B}$$