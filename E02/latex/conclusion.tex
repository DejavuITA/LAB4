\subsection{Conclusioni}
In questa esperienza abbiamo potuto osservare come gli OPAMP, sebbene siano dei circuiti abbastanza precisi, abbiano delle imperfezioni, date dalla loro composizione circuitale (sono presenti dei transistor BJT al loro interno).
Le discrepanze tra il modello ideale e l'OPAMP reale sono date principalmente dallo sbilanciamento della risposta dello stesso ($V_{offset}$) e dalle correnti di polarizzazione (\textit{bias currents}).
Per nostra fortuna spesso gli OPAMP presentano dei connettori predisposti a minimizzare la tensione di offset con circuiti di compensazione: nel nostro caso un trigger collegato ai piedini di offset e all'alimentazione negativa.
Una volta bilanciato l'opamp, abbiamo misurato le correnti di polarizzazione e abbiamo potuto osservare che esse sono dell'ordine dei \si{\nano\ampere}, quindi trascurabili per gli utilizzi più comuni.
