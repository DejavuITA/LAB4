\section{11.11.2014 - Porte logiche}

In questa esperienza verificheremo il funzionamento di alune porte logiche TTL e semplici circuiti costruiti con esse.

\subsection*{Strumenti e materiali}

\begin{itemize} [noitemsep]
	\item Oscilloscopio Agilent DSO-X 2002A (bandwidth \SI{70}{\mega\hertz}, sample rate \num{2} GSa/s);
	\item Generatore di tensione continua Agilent E3631A (max $\pm \, \SI{25}{\volt}$ o $\pm \, \SI{6}{\volt}$);
	\item Multimetro Agilent 34410A a sei cifre e mezza;
	\item Un integrato 7400, composto di quattro porte logiche TTL NAND; % '00;
	\item Basetta a LED;		
	\item Resistenze e capacità di vari valori;
	\item Breadboard e cablaggi vari.
\end{itemize}

\subsection{Logiche TTL e scheda di visualizzazione}

Le porte logiche a tecnologia TTL funzionano utilizzando come 1 logico il valore di tensione +\SI{5}{\volt} mentre come 0 logico \SI{0}{\volt}.
Pertanto l'alimentazione che porteremo alla breadboard non sarà più $\pm$\SI{15}{\V} e GND, ma solo $+$\SI{5}{\V} e GND.
Ricordiamo che l'alimentazione deve essere stabilizzata e disaccoppiata con i condensatori come nel caso analogico in quanto nei momenti di commutazione delle porte abbiamo grandi variazioni di tensione in piccoli tempi (idealmente la variazione da stato 1 a 0 è istantanea).
È dunque essenziale avere una riserva di cariche utilizzabili dal circuito nel momento in cui si hanno commutazioni.

Per controllare il funzionamento dei nostri circuiti utilizzeremo una schedina di visualizzazione a led fornitaci già assemblata.
Essa permette di visualizzare lo stato "alto" o "basso" delle linee da monitorare tramite 8 led (4 rossi e 4 verdi).
Tali schede contengono un integrato buffer 74LS244 nella logica TTL che fornisce in uscita una corrente di \SI{-15}{\milli\ampere} per il livello alto e di \SI{24}{\milli\ampere} per il livello basso.

\subsubsection{Porta NAND}

Abbiamo verificato il funzionamento di una porta NAND utilizzando, come già detto, una scheda di visualizzazione a LED.
È stata verificata la seguente tavola di verità.\\

\begin{figure}[htpc]
\centering
	\begin{subfigure}[hc]{.4\textwidth}
		\centering
		{\renewcommand{\arraystretch}{1.2}%
		\begin{tabular}{|c|c|c|}
		\hline
		A & B & $\overline{AB}$ \\
		\hline
		0 & 0 & 1\\
		\hline
		0 & 1 & 1\\
		\hline
		1 & 0 & 1\\
		\hline
		1 & 1 & 0\\
		\hline
		\end{tabular}}
		\label{tab9:NAND}
        \end{subfigure}
        \begin{subfigure}[hc]{.4\textwidth}
		\centering
		\includegraphics[width=.35\textwidth]{../E09/latex/NAND.pdf}
		\label{cir9:nand}
        \end{subfigure}
\caption{Tabella di verità e simbolo circuitale convenzionalmente utilizzato per porte NAND.}
\end{figure}

\subsubsection{Porta NOT}

\begin{wrapfigure}[10]{r}{0.4\textwidth}
\centering
\includegraphics[width=.16\textwidth]{../E09/latex/NOT.pdf}
\caption{Implementazione e simbolo circuitale per porte NOT.}
\label{cir9:not}
\end{wrapfigure}

Non avendo a disposizione una porta NOT, abbiamo utilizzato una porta NAND cortocircuitando i due ingressi tra loro.
Così facendo otteniamo un'implementazione di una porta NOT.

Ne abbiamo ininzialmente verificato il funzionamento collengandola alla scheda di visualizzazione led e fornendo un'onda quadra in ingresso ($V_{pp}=\SI{5}{\volt}$, $V_{off}=+\SI{2.5}{\volt}$, $\nu=\SI{2}{\hertz}$).
Una volta eseguito questo test preliminare, abbiamo collegato ingresso ed uscita all'oscilloscopio e ne abbiamo analizzato la risposta per diverse tensioni costanti in ingresso.
I dati ottenuti sono riportati in Tabella \ref{tab9:risposta2}.
Come vediamo, tra \SI{1}{\volt} e \SI{1.5}{\volt} la tensione varia in modo molto rapido.

\vspace{2mm}
\begin{table}[htpc]
\centering
{\renewcommand{\arraystretch}{1.1}%
\begin{tabular}{|l|c|c|c|c|c|c|c|c|c|c|c|}
\hline
$V_{in}$ [\si{\volt}] & 0.0 & 0.5 & 1.0 & 1.5 & 2.0 & 2.5 & 3.0 & 3.5 & 4.0 & 4.5 & 5.0 \\
\hline
$V_{out}$ [\si{\volt}] & 4.30 & 4.05 & 3.01 & 0.10 & 0.10 & 0.10 & 0.10 & 0.10 & 0.10 & 0.10 & 0.10 \\
\hline
\end{tabular}}
\caption{I valori sono stati ricavati variando la tensione in entrata di \SI{.5}{\V}.}
\label{tab9:risposta2}
\end{table}

Siamo dunque interessati a capire meglio la risposta della porta in funzione della tensione in ingresso.
Per fare ciò utilizziamo la modalità XY dell'oscilloscopio.

Abbiamo dunque fornito un'onda sinusoidale all'ingresso di frequenza \SI{100}{\kilo\hertz}, $V_{pp}=\SI{5}{\volt}$ e $V_{off}=+\SI{2.5}{\volt}$.
Come vediamo dal grafico sotto riportato, sebbene stiamo lavorando con segnali digitali (1 o 0) le tensioni in gioco sono manifestamente analogiche.
Avremo dunque tutta la serie di problemi già affrontati nella parte analogica del corso.
Vediamo infatti che la tensione in uscita non è una funzione a gradino.
Inoltre, per evitare che rumori disturbino il segnale in uscita, avremo bisogno di una sorta di isteresi.
Non esisterà dunque un valore ben definito di V alta e V bassa, ma delle bande di lavoro standard per la classe di circuiteria scelta (nel nostro caso TTL), riportate solitamente nel data-sheet.
Lo standard TTL è riportato in Figura \ref{fig9:TTL}

\begin{figure}[htpc]
\centering
	\begin{subfigure}[hc]{0.49\textwidth}
		\centering
		\includegraphics[width=.95\textwidth]{../E09/latex/XY.pdf}
                \caption{Risposta in uscita di una porta NAND con entrate cortocircuitate (= NOT) ad un'onda triangolare di frequenza \SI{100}{\kHz} in entrata.}
                \label{fig9:XY}
        \end{subfigure}%
	\quad
        \begin{subfigure}[hc]{0.49\textwidth}
		\centering
		\includegraphics[width=.95\textwidth]{../E09/latex/TTL.png}
                \caption{Livelli standard di input e output per la logica TTL.}
                \label{fig9:TTL}
        \end{subfigure}
\caption{}
\end{figure}
\vspace{-5mm}

\subsubsection{Circuito di GATE con porta AND}

Per realizzare il circuito GATE è sufficiente utilizzare una porta AND con un segnale di controllo ad un ingresso.
Per realizzare una porta AND possiamo utilizzare 2 NAND come proposto nella seguente figura.
Infatti, utilizzando la notazione convenzionale, $AB=\overline {\overline {AB}}$ che trivialmente è una NAND e una NOT in serie.

La tavola di verità è la seguente (A=C=controllo, B=S=segnale):

\begin{figure}[htpc]
\centering
	\begin{subfigure}[hc]{.4\textwidth}
		\centering
		{\renewcommand{\arraystretch}{1.1}%
		\begin{tabular}{|l|l|c|}
		\hline
		C & S & CS \\
		\hline
		0 & 0 & 0\\
		\hline
		0 & 1 & 0\\
		\hline
		1 & 0 & 0\\
		\hline
		1 & 1 & 1\\
		\hline
		\end{tabular}}
		\label{tab9:AND}
        \end{subfigure}
        \begin{subfigure}[hc]{.4\textwidth}
		\centering
		\includegraphics[width=.5\textwidth]{../E09/latex/AND.pdf}
		\label{cir9:AND}
        \end{subfigure}
\caption{Tabella di verità e simbolo circuitale delle porte AND e implementazione con porte NAND}
\end{figure}

Come vediamo dalla tabella, se il segnale di controllo è a 0 logico l'uscita sarà sempre 0 logico.
Se invece il controllo è a 1 logico, allora l'uscita sarà una copia del segnale in ingresso.
Per verificare empiricamente ciò, abbiamo collegato l'oscilloscopio all'uscita del circuito e fornito un'onda quadra ($V_{pp}=\SI{5}{\volt}$, $V_{off}=+\SI{2.5}{\volt}$, $\nu=\SI{2}{\hertz}$).
Abbiamo visto che quando il controllo era collegato a comune, l'uscita era praticamente nulla (\num{40} -- \SI{60}{\mV}).
Con il controllo a 0 logico, abbiamo osservato invece la stessa onda quadra posta in ingresso anche in uscita.

\subsubsection{Porta XOR}

Una porta XOR implementa la funzione matematica della somma diretta, cioè segue la tabella di verità in Figura \ref{cir9:xor}. Noi abbiamo a disposizione solo porte NAND, dobbiamo cercare quindi cercare di progettare un circuito equivalente alla XOR, cioè che rispetti tale tabella di verità.

\begin{figure}[htpc]
\centering
	\begin{subfigure}[hc]{.4\textwidth}
		\centering
		{\renewcommand{\arraystretch}{1.1}%
		\begin{tabular}{|c|c|c|}
		\hline
		A & B & $A \oplus B$ \\
		\hline
		0 & 0 & 0\\
		\hline
		0 & 1 & 1\\
		\hline
		1 & 0 & 1\\
		\hline
		1 & 1 & 0\\
		\hline
		\end{tabular}}
		\label{tab9:XOR}
        \end{subfigure}
        \begin{subfigure}[hc]{.15\textwidth}
		\centering
		\includegraphics[width=.99\textwidth]{../E09/latex/XOR.pdf}
%		\caption{Implementazione e simbolo circuitale per porte XOR.}
		\label{cir9:XOR}
	\end{subfigure}
        \begin{subfigure}[hc]{.4\textwidth}
		\centering
		\includegraphics[width=.7\textwidth]{../E09/latex/iXOR.pdf}
%		\caption{Implementazione e simbolo circuitale per porte XOR.}
		\label{cir9:iXOR}
        \end{subfigure}
\caption{Tabella di verità e simbolo circuitale delle porte XOR e implementazione con sole porte NAND}
\label{cir9:xor}
\end{figure}

Senza utilizzare le mappe di Karnaugh, è facile vedere che vale $A \oplus B=A\overline B + \overline A B$ e applicando il teorema di De Morgan si ottiene:

$$A \oplus B=A\overline B + \overline A B=\overline{\overline{A\overline B + \overline A B}}=\overline{\overline{A\overline B} \cdot \overline{\overline A B}}$$

Dunque, il circuito per realizzare una porta XOR usando porte NAND e NOT è quello riportato in Figura \ref{cir9:xor}: la realizzazione ha richiesto 2 porte NOT e 3 NAND. Ne abbiamo infine verificato il funzionamento con la scheda a LED.

\subsection{Votazione con 3 Giurati e 1 Presidente}

Consideriamo un'assemblea composta da tre giurati e un presidente il cui compito è votare a favore o contro una proposta, in cui il voto del presidente vale il doppio.
Il circuito che restituisce l'esito della votazione prevede di sommare i voti dei vari membri tenendo però in considerazione che il voto del presidente ha peso doppio.
Se la maggioranza dei voti ($\geq 50\%$) è positiva allora la votazione ha esito positivo, altrimenti ha esito negativo.

Partiamo dunque dalla mappa di Karnaugh in Tabella \ref{tab9:giurati}. Minimizzandola otteniamo

\begin{wraptable}[7]{c}{.4\textwidth}
\centering
{\renewcommand{\arraystretch}{1}%
\begin{tabular}{|c|c|c|c|c|}
\hline
\diaghead{\theadfont lololololo a} {CP}{AB}& 00 & 01 & 11 & 10\\
\hline
00 & 0 & 0 & 0 & 0 \\
\hline
01 & 0 & 1 & 1 & 1 \\
\hline
11 & 1 & 1 & 1 & 1 \\
\hline
10 & 0 & 0 & 1 & 0 \\
\hline
\end{tabular}}
\caption{}
\label{tab9:giurati}
\end{wraptable}

$$Y=ABC+P(A+B+C)$$

Come fatto prima, avendo a disposizione solo porte NAND, trasformiamo tutto in prodotti.

\vspace{-1mm}
\begin{minipage}{0.6\textwidth}
\begin{align}
Y 	&= ABC+P(A+B+C)
	= \overline{\overline{ABC+P(A+B+C)}} \nonumber \\
	&= \overline{\overline{ABC} \cdot \overline {P(A+B+C)} }
	= \overline{\overline{(AB)C} \cdot \overline {P(\overline{\overline{{A+B+C}} })}} \nonumber \\
	&= \overline{\overline{(AB)C} \cdot \overline {P(\overline{{(\overline A \cdot \overline B) \cdot \overline C} })}} \nonumber
\end{align}
\end{minipage}
\vspace{3mm}

Ora è immediato costruire il circuito utilizzando porte NAND e NOT.
Lo schema è riportato in Figura \ref{cir9:giudici}.\\
Come per gli altri circuiti, ne è stato verificato utilizzando la schedina a LED. 

\begin{figure}[htpc]
\centering
\includegraphics[width=.75\textwidth]{../E09/latex/giudici.pdf}
\caption{Circuito logico dedicato alla votazione di una assemblea collegiale composta da un presidente e tre giurati.}
\label{cir9:giudici}
\end{figure}

\subsection{Allarme per appartamento}

In quest'ultima parte dell'esperienza cercheremo di progettare e costruire un circuito di allarme. Immaginiamo dunque di avere un mini-appartamento in cui sono presenti sensori d'apertura su porte e finestre e un sensore di movimento: vogliamo analizzarne i segnali per decidere se fare suonare l'allarme.

Chiamiamo P il sensore sulla porta (0=chiusa), F il sensore sulla finestra (0=chiusa), I il sensore ad infrarossi (0=no movimento) e infine C la chiave che permette di attivare l'allarme escludendo il sensore ad infrarossi (1=escludo infrarossi).

\begin{wraptable}[9]{c}{.4\textwidth}%[htpc]
%\begin{minipage}{0.6\textwidth}
\centering
{\renewcommand{\arraystretch}{1}%
\begin{tabular}{|c|c|c|c|c|}
\hline
\diaghead{\theadfont lololololo a} {IC}{PF}& 00& 01 & 11&10\\
\hline
00 & 0 & 1 & 1 & 1 \\
\hline
01 & 0 & 1 & 1 & 1 \\
\hline
11 & 0 & 1 & 1 & 1 \\
\hline
10 & 1 & 1 & 1 & 1\\
\hline
\end{tabular}}
\caption{}
\label{tab9:allarme}
%\end{minipage}
\end{wraptable}

Osserviamo che uno dei possibili raggruppamenti dati dalla mappa di Karnaugh, riportata in Tabella \ref{tab9:allarme} è
$$Y=P+F+\overline C I$$
Avendo a disposizione solo porte NAND trasformiamo l'espressione in sole moltiplicazioni con il teorema di De Morgan.

$$Y=P+F+\overline C I=\overline{\overline{P+F+\overline C I}}=\overline{(\overline P \cdot \overline F) \cdot \overline{\overline C I}}$$ 

Il circuito risulta quello nella seguente figura.
È stato verificato come gli altri utilizzando la scheda a LED.

\begin{figure}[htpc]
\centering
\includegraphics[width=.65\textwidth]{../E09/latex/allarme.pdf}
\caption{Circuito logico dedicato alla gestione dell'allarme in un mini-appartamento.}
\label{cir9:allarme}
\end{figure}

\subsection*{Conclusioni}
In questa esperienza abbiamo verificato il funzionamento di alcune porte, evidenziando il fatto che sebbene i segnali trattati siano digitali (0 e 1 logico, 0 e $+$\SI{5}{\V}), non possiamo trascurare la tecnologia analogica che sta alla base del funzionamento.
Inoltre, abbiamo visto come con l'utilizzo delle più comuni porte NAND, possiamo implementare le altre porte.
Infine, utilizzando le proprierà algebriche della logica booleana, abbiamo semplificato circuiti complessi e li abbiamo implementati con il solo utilizzo di porte NAND.