\subsection{Premesse}

Per progettare un circuito che legga il segnale elettrico del nostro battito cardiaco a livello della cute, dobbiamo considerare che:
\begin{itemize} [noitemsep]
	\item Sulla superficie del corpo il segnale ECG è normalmente di circa \SI{1}{mV} (al massimo raggiunge i \SI{4}{mV}) ed ha una frequenza di circa \SI{1.25}{\Hz}.
	\item Gli elettrodi a nostra disposizione, che realizzano il collegamento tra la strumentazione e il corpo, creano un potenziale di circa \SI{700}{\mV}, quindi circa \num{200} volte maggiore del segnale cardiaco.
	\item Il segnale da misurare è piccolo, quindi dobbiamo porre particolare attenzione alle fonti di rumore elettromagnetico dati da effetti capacitivi e/o induttivi. Inoltre dobbiamo evitare che i cavi e il paziente si muovano durante la misurazione.
	\item È necessario che il paziente sia isolato dal lato di elaborazione del segnale, alimentato dall'alimentatore da banco, per evitare problemi di giri di massa.
\end{itemize}

Tenendo conto di questi aspetti, nei paragrafi successivi analizzeremo gli aspetti legati all'abbattimento dei rumori e all'amplificazione del segnale in ingresso.

\subsection{Abbattimento dei rumori}

\subsubsection*{Operazionale U1}

Con il primo stadio in entrata (operazionale U1 in Figura \ref{cir8:compensation}) eliminiamo i segnali di modo comune generati dagli elettrodi (oltre ad avere un'alta impedenza in ingresso e quindi un valore in entrata ad U1 più fedele alla reale ddp generata dal corpo). In testa ed in coda a questo stadio sono anche presenti delle capacità che creano due filtri: rispettivamente un passa basso ed un passa alto.

\paragraph*{Passa Basso}
Posto in testa ad U1, e composto dalle capacità $C_c$ e $C_d$, è necessario per abbattere eventuali rumori ambientali. Inserendo le capacito nel circuito come in Figura \ref{cir8:compensation}, il datasheet dell'operazionale ci fornisce direttamente le frequenze di taglio di modo differenziale
\begin{equation*}
	\frac{1}{2 \pi R ( C_d + C_c ) } = \SI{194}{\Hz}
\end{equation*}
e di modo comune
\begin{equation*}
	\frac{1}{2 \pi R C_c} = \SI{4}{\kHz}
\end{equation*}

Con queste capacità riusciamo dunque ad eliminare i segnali di modo comune ad alte frequenze, ma soprattutto tagliamo ogni frequenza superiore ai \SI{200}{\Hz} in modo differenziale (il nostro segnale è infatti sui \SI{1.25}{\Hz} come da premesse), e dunque restringiamo drasticamente le frequenze ammesse ad essere amplificate dal nostro circuito.

\paragraph*{Passa Alto}
Essendo posto all'uscita dell'operazionale (capacità $C_1$ in Figura \ref{cir8:compensation}, elimina eventuali segnali continui in output. La sua frequenza di taglio è di circa \SI{3}{\Hz}. Inoltre, è necessario che l'impedenza siano adattate, ovvero dovrà essere vista come infinita in entrata al prossimo blocco: per far ciò poniamo un ulteriore operazionale (un OP07, U2 in Figura \ref{cir8:compensation}).

\subsubsection*{Active gards}

Polarizzo la calza del cavo coassiale che connette l’elettrodo all’amplificatore ad un potenziale pari alla tensione di modo comune.
In tal modo riduco l’effetto delle capacità parassite, e aumento la reiezione di modo comune.
\begin{itemize}
\item il cavo che porta il segnale “vede” una differenza di potenziale molto bassa. Limito le perdite nell’isolante
\item entrambi i cavi “vedono” lo stesso potenziale, indipendentemente dalla loro posizione sul banco)
\end{itemize}

\subsubsection*{Amplificatore di isolamento}
Isola galvanicamente i 2 lati del circuito: quello collegato al paziente ed alimentato a batteria e quello di uscita ed elaborazione del segnale alimentato dall’alimentatore da banco.
\begin{itemize}
	\item Isola fino a 1500 volt -Alimentazione da+/-4,5V a +/-18V.
	\item Possiede 2 alimentazioni separate e quindi anche due riferimenti di massa separati.
	\item Non richiede componenti esterni (tranne le capacità di disaccoppiamento)
\end{itemize}

\subsection{Amplificazione}

Dato il basso valore di tensione del segnale in entrata, abbiamo anche la necessità di amplificarlo per permettere alla strumentazione di leggerlo. Nel circuito ci sono tre stadi di amplificazione che amplficano il segnale di circa 800 volte.

\paragraph*{Operazionale U1}
L'AD622, come visto nell'Esperienza 5, controlla l'amplificazione tramite la presenza di una resistenza posta nei punti dedicati della piedinatura (si faccia riferimento alla Figura \ref{cir5:ad622_ponte}). Con le resistenze utilizzate, l'amplificazione è data dalla legge (\ref{eq5:guadagno_AD622}) fornitaci dal costruttore
$$G_1=1+\frac{50.5 \si{\kilo\ohm}}{R_g} \approx 116$$

\paragraph*{Filtro passa alto}
Con la sua frequenza di taglio di circa \SI{3}{\Hz}, il filtro passa alto in coda ad U1 riduce il nostro segnale (in uscita dal primo stadio di amplificazione dato da U1). Possiamo stimare di quanto viene ridotto considerando la funzione di partizione di un passa alto e facendone il modulo, ottenendo il guadagno. Vale dunque che
$$G_2=\frac{2 \pi R_1 C_1 \times \SI{1.25}{\Hz}}{\sqrt{1+(2 \pi R_1 C_1 \times \SI{1.25}{\Hz})^2}}\approx 0.35$$

\paragraph*{Operazionale U3}
Questo operazionale è posto in un blocco con retroazione negativa non invertente. Dunque il suo guadagno è dato da
$$G_3=1+\frac{R_2}{R_3} \approx 11$$

L'amplificazione totale è dunque
$$G = G_1 \times G_2 \times G_3 \approx 450$$

\begin{figure}[tpc]
\centering
\includegraphics[width=.7\textwidth]{../E07/latex/circuito.pdf}
\caption{Schema del circuito da noi utilizzato per acquisire l'elettrocardiogramma. Il diverso colore della comune serve a rendere evidente l'indipendenza delle comuni fra il circuito e il collegamento all'oscilloscopio.}
\label{cir8:compensation}
\end{figure}