\subsection*{Tempo di propagazione}
In questa prima parte dell'esperienza analizzeremo il PDT (Propagation Delay Time) di una porta NAND. Tale parametro è il ritardo con cui l'uscita commuta rispetto all'istante in cui commuta l'ingresso. Poichè tale intervallo temporale è di pochi \si{\nano\second}, è stato deciso di collegare in serie 3 porte NOT. Così facendo il ritardo viene amplificato di 3 volte. Possiamo ora costruire il circuito riportato in Fig.(). Come vediamo, se non esistesse il ritardo l'uscita sarebbe sempre ad 1 logico ($\approx 5 \si{\volt}$). A causa del ritardo, però, quando il segnale in ingresso commuta da 0 ad 1, la serie di porte NOT non riuscità instantaneamente a passare da 1 a 0 e dunque, per un istante
