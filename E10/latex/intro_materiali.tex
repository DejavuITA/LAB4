\section{19.11.2014 - Porte logiche II}

In questa esperienza studieremo le porte logiche TTL in maniera più approfondita e ne cercheremo di capire le problematiche. Analizzeremo porte tri-state e costruiremo un circuito di multiplexing.

\subsection*{Strumenti e materiali}

\begin{itemize} [noitemsep]
	\item Oscilloscopio Agilent DSO-X 2002A (bandwidth \SI{70}{\mega\hertz}, sample rate \num{2} GSa/s);
	\item Generatore di tensione continua Agilent E3631A (max $\pm \, \SI{25}{\volt}$ o $\pm \, \SI{6}{\volt}$);
	\item Multimetro Agilent 34410A a sei cifre e mezza;
	\item Un integrato 7400, composto di quattro porte logiche TTL NAND; % '00;
	\item Un integrato 74LS05, composto di quattro porte logiche NOT TTL Open Collector;
	\item Un integrato 74LS125, composto di quattro porte logiche buffer TTL 3State;
	\item Basetta a LED;		
	\item Resistenze e capacità di vari valori;
	\item Breadboard e cablaggi vari.
\end{itemize}
