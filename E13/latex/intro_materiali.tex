\section{9.12.2014 - Convertitore analogico-digitale}

In questa esperienza studieremo il funzionamento, procedendo anche al montaggio, di un convertitore da analogico a digitale.

\subsection*{Strumenti e materiali}

\begin{itemize} [noitemsep]
	\item Oscilloscopio Agilent DSO-X 2002A (bandwidth \SI{70}{\mega\hertz}, sample rate \num{2} GSa/s);
	\item Generatore di tensione continua Agilent E3631A (max $\pm \, \SI{25}{\volt}$ o $\pm \, \SI{6}{\volt}$);
	\item Multimetro Agilent 34410A a sei cifre e mezza;
	\item Circuito montato e valutato nell'esperienza precedente (si veda il Convertitore digitale-analogico in Relazione \ref{rel12:label});
%	\item Un integrato 7400, composto di quattro porte logiche TTL NAND; % '00;
%	\item Un integrato 74LS02, composto di quattro porte logiche NOR;
%	\item Un integrato 74LS109;
%	\item Un integrato 74LS191;
\item Un integrato LM311;	
%	%\item Un integrato 74LS125, composto di quattro porte logiche buffer TTL 3State;
%	\item Basetta a LED;		
\item Resistenze (trimmer e non) aggiuntive di vari valori;
\item Breadboard e cablaggi vari.
\end{itemize}
