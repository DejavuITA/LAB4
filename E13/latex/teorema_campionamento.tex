\subsection{Teorema del campionamento}

Il teorema del campionamento (o teorema di Nyquist-Shannon) è enunciato come segue. Per effettuare il campionamento di un segnale senza perdita di informazione ed evitando il fenomeno dell'aliasing, la frequenza di campionamento $f_c$ deve essere il doppio della frequenza massima $f_s$ del segnale da campionare.

Bisogna però tenere presente che tale teorema è valido solo se la trasformata di Fourier del segnale da campionare è limitata nel dominio delle frequenze, cioè esiste solo una banda di frequenze in cui la trasformata non è nulla.