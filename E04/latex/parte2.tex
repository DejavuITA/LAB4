\subsection{Trigger di Schmidt non-invertente}

Quando nel segnale è presente un rumore è possibile che il valore di tensione restituito dal comparatore cambi più volte da $0$ a $V_{CC}$. Infatti, l'ampiezza in tensione del rumore può essere tale da passare sopra e sotto la tensione di soglia, invertendo la $V_{out}$ più volte secondo la \ref{eq4:comparatore}. Per ovviare a questo problema si sfrutta la retroazione positiva dell'amplificatore operazionale per riportare parte della $V_{out}$ sull'ingresso non invertente: ciò permette di creare un offset positivo sulla tensione all'ingresso non invertente quando $V_{in}>0$ e viceversa. Quindi, durante il confronto del segnale con la tensione di riferimento, il segnale in entrata risulterà aumentato o diminuito di una certa quantità a seconda di $V_{out}$.

Valutiamo queste tensioni di soglia nel trigger di Schmidt non invertente (grafico in Figura ????). Per fare ciò cerchiamo la dipendenza di $V_{in}$ da $V_{out}$ e da $V^+$ (tensione all'ingresso non invertente). Utilizzando la legge di Kirkhhoff sul nodo dell'ingresso non invertente
$$\frac{V_{in}-V^+}{R_1} + \frac{V_{out}-V^+}{R_2} = 0$$
da cui otteniamo
\begin{equation}
V_{in} = V^+ \frac{R_1+R_2}{R_2} - V_{out} \frac{R_1}{R_2}
\label{eq4:v_in_parte2}
\end{equation}
Dunque otteniamo i valori di soglia (che sono valori di $V_{in}$) ponendo la tensione $V^+=V^-=V_{ref}=0$, cioè il punto di tensione di inversione del comportamento del 311 a collettore comune, e considerando i due valori possibili di $V_{out}$ dati dalla (\ref{eq4:comparatore}), cioè $V_{out}^{inf}=0$ e $V_{out}^{sup}=V_{CC}=15$\si{\volt}):
$$V_{soglia}^{sup} = - V_{out}^{inf} \frac{R_1}{R_2} = 0 \qquad V_{soglia}^{inf} = - V_{out}^{sup} \frac{R_1}{R_2} = 1.5 \si{\volt}$$
da cui è possibile definire la tensione di \textit{isteresi} (cioè la tensione massima in cui un segnale può oscillare attorno alla tensione di riferimento prima di ribaltare l'output del 331)
$$V_{isteresi} = |\Delta V| = |V_{soglia}^{sup} - V_{soglia}^{inf}| = V_{out}^{sup} \frac{R_1}{R_2} = 1.5 \si{\volt}$$
I punti così calcolati sono visibili nel grafico di isteresi in Figura ??????. Notiamo inoltre che le rette $V_{in}=V_{soglia}$ (per entrambe le soglie) in realtà non sono perpendicolari alle ascisse come si vede sui grafici ideali: abbiamo ipotizzato che tale comportamento sia imputabile all'entrata dell'opamp in regione lineare, uscendo dalla saturazione.

%e amplifica la tensione in ingresso con
%$$V_{out} = -\frac{R_2}{R_1} V_{in}$$
%dove tale relazione è ricavata da (\ref{eq4:v_in_parte2}) nell'intorno di $V^+=0$. %CONTROLLARE DAL GRAFICO!!!!

Abbiamo inoltre provato ad inserire un offset all'ingresso invertente: ciò ha portato, come ci aspettavamo dalla relazione (\ref{eq4:v_in_parte2}) (si ricorda che $V^+=V_{ref}$), ad una traslazione del grafico in Figura ?????? di un valore dato dall'offset impostato. Infine, durante l'esperienza, abbiamo anche variato $R_1$ per valutare diverse tensioni di isteresi.