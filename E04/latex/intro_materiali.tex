\section{08.10.2014 - Comparatori}

In questa esperienza studieremo dei circuiti comparatori con retroazione positiva. In particolare, analizzeremo il funzionamento del \textit{trigger di Schmitt} non invertente, dell'oscillatore a rilassamento e di un interruttore crepuscolare. 


\subsection*{Strumenti e materiali}

\begin{itemize} [noitemsep]
\item Oscilloscopio Agilent DSO-X 2002A (bandwidth \SI{70}{\mega\hertz}, sample rate \num{2} GSa/s);
\item Generatore di tensione continua Agilent E3631A (max $\pm \, \SI{25}{\volt}$ o $\pm \, \SI{6}{\volt}$);
\item Generatore di forme d'onta Agilent 33120A con range di frequenza da \SI{100}{\micro\hertz} a \SI{15}{\mega\hertz};
\item Multimetro Agilent 34410A a sei cifre e mezza;
%\item Un amplificatore operazionale $\mu$A741;
\item Un amplificatore operazionale LM311;
\item Fototransistor OP550;
\item Resistenze e capacità di vari valori;
\item un trimmer a un giro da \SI{5}{\kilo\ohm} e uno da \SI{10}{\kilo\ohm};
\item un trimmer multigiro da \SI{10}{\kilo\ohm};
\item Breadboard e cablaggi vari.
\end{itemize}
