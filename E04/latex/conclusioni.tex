\subsection*{Conclusioni}

In questa esperienza abbiamo analizzato il funzionamento di circuiti comparatori, molto utili quando vogliamo confrontare due segnali ottenendo un semplice valore digitale (on-off). Tuttavia, vista la sensibilità degli op-amp, abbiamo bisogno di un modo per evitare che il rumore influenzi le nostre comparazioni. Per far questo abbiamo introdotto i circuiti con isteresi, ovvero con feedback-positivo. \'E stato poi realizzato un interruttore crepuscolare. Questi interruttori, contenenti un fototransistor, permettono ad esempio l'accensione (e spegnimento) delle luci stradali quando tramonta il sole. Non si vuole però che una semplice nuvola passeggera faccia accendere le luci. Bisogna dunque introdurre un' insteresi per far sì che il rumore (la nuvola) non faccia scattare l'interruttore. Infine, abbiamo costruito un oscillatore a rilassamento. Peculiarità di tale circuito è che esso oscilla tra $+V_{sat}$ e $-V_{sat}$ sebbene sia alimentato con tensioni continue. La frequenza di tale oscillatore è univocamente determinata dai valori delle componenti circuitali.
