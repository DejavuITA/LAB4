\subsection{Registro a scorrimento ad anello}
La realizzazione di un registro a scorrimento a ricircolazione prevede l'utilizza di FF J-K in modalità D. Tale circuito, come visto a lezione, permette di trasmettere ad ogni fronte di salita del Clock il segnale memorizzato in un FF al successivo. Essendo l'ultimo collegato al primo, il segnale continua a circolare nel nostro circuito. Proprio per questo tale registro viene chiamato ad anello. Lo schema circuitale è riportato in figura. 

Per memorizzare gli 1 e 0 logici utilizziamo le linee di Preset e Clear presenti nei FF. Memorizzando inizialmente un 1 nel primo FF e 0 in tutti gli altri, ad ogni ciclo di Clock il nostro 1 passerà al successivo e così facendo avremo un segnale alto che cicla all'infinito. Abbiamo dunque verificato tale circuito con la solita schedina al led. 



\subsection{Contatore up/down sincrono}



\subsection{Convertitore Digitale-Analogico}
