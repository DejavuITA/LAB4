\subsection{Registro a scorrimento ad anello}
La realizzazione di un registro a scorrimento a ricircolazione prevede l'utilizza di FF J-K in modalità D. Tale circuito, come visto a lezione, permette di trasmettere ad ogni fronte di salita del Clock il segnale memorizzato in un FF al successivo. Essendo l'ultimo collegato al primo, il segnale continua a circolare nel nostro circuito. Proprio per questo tale registro viene chiamato ad anello. Lo schema circuitale è riportato in figura. 

Per memorizzare gli 1 e 0 logici utilizziamo le linee di Preset e Clear presenti nei FF. La linea RC connessa al Preset del nostro primo FF e ai Clear di tutti gli altri permette di caricare inizialmente un 1 nel primo FF e 0 in tutti gli altri. Ad ogni ciclo di Clock il nostro 1 passerà al successivo e così facendo avremo un segnale alto che cicla all'infinito.

Il motivo per cui è stato messa una linea RC è quella di permettere un set automatico del nostro dispositivo all'accensione. Infatti il nostro generatore di tensione impiegherà un certo $\delta t$ a portarsi a regime (\SI{5}{\volt}). A $t=0^+$ la tensione $V_c$ sarà zero e dunque setterà i nostri bit con un 1 nel primo FF e 0 negli altri (ricordiamo che Clear e Preset sono attivi bassi). Ma essendoci un RC, la tensione in $V_c$ cresce più lentamente della tensione fornita dal generatore. Dopo un tempo adeguato in base al circuito RC (solitamente è sufficiente $4\tau$) la nostra tensione $V_c$ sarà praticamente la tensione di alimentazione e dunque avremo disattivate il Clear e Preset. Dovremo dunque attendere un tempo pari a circa $4\tau$ prima di inviare un segnale di Clock. Dai nostri calcoli risulta che il $\tau$ del nostro circuito è circa $\SI{545}{\milli\second}$ da cui otteniamo che il tempo che dobbiamo attendere è di circa $\SI{545}{\milli\second}$.
% Se accendiamo il Clock dopo aver alimentato il nostro circuito dunque siamo sicuri che i bit sono memorizzati correttamente.



Abbiamo dunque verificato tale circuito con la solita schedina al led. 



\subsection{Contatore up/down sincrono}

Il contatore up/down è realizzato utilizzando 4 FF JK Master-Slave presenti un un integrato 74LS191. Ricordiamo che tutti i FF sono sincronizzati sul fronte di salita del Clock. Tale integrato dà la possibilità di contare in avanti o indietro in base al controllo Up/Down (tale impostazione può essere fatta solo ed esclusivamente quando il Clock è alto). Inoltre è possibile settare un valore inziale indipendentemente dal Clock. Ciò è particolarmente comodo se su vuole partire a contare da un determinato numero. Il nostro contatore così fatto è a 4 bit. Ciò significa che possiamo contare da 0 a 15. Esiste tuttavia una linea (/CE) che permette di gestire il prestito/riporto in contatori a più stadi in cascata e altre due linee (TC e /RC) che permettono di indicare situazioni di overflow/underflow. La linea TC è normalmente a 0 e va ad 1 logico quando il contatore raggiunge 0 in count-down o 15 in count-up. La linea /RC è normalmente a 1 logico e quando CE=0 e TC=1 allora /RC va a 0 e rimane in tale stato fino a quando il Clock non ritorna alto.

Con tali linee in modo appropriato possiamo contare a 8 bit utilizzando due integrati 74LS191 e semplicemente collegarli utilizzando l'ingresso.

Possiamo ora scegliere due metodi per implementare il contatore: asincrono o sincrono. Come già visto a lezione, il contatore asincrono ha il priblema che ogni FF passa il Clock al successivo, passando dunque i ritardi che sono presenti. Il metodo sincrono invece non ha problemi di ritardi in quanto tutti i FF sono sincronizzati dallo stesso Clock. 

Noi costruiremo dunque un contatore sincrono. Lo schema circuitale è riportato nella seguente figura.

$$figura$$

Come vediamo abbiamo messo una linea RC come nel caso del circuito precedente per settare in automatico il valore iniziale (in questo caso abbiamo settato 0). Abbiamo inoltre inserito un FF di tipo D sincronizzato con il Clock del contatore. Tale elemento,  collegato agli ingressi Up/Down, fa si che se decidiamo di cambiare il controllo Up/Down il segnale dovrà passare attraverso il FF di "controllo". Dato che abbiamo i ritardi, tale segnale raggiungerà il contatore solo  quando il Clock è alto. Con tale accorgimento non rischiamo di interferire con il contatore quando il Clock è basso, cosa che potrebbe provocare errori nel conteggio.





Ne abbiamo verificato il funzionamento utilizzando la basetta al led, provando a contare sia indietro che in avanti. 





\subsection{Convertitore Digitale-Analogico}




