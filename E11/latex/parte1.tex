\subsection{Flip-Flop}

\subsubsection*{FF SR con porte NAND tipo latch}

In questa prima parte dell'esperienza abbiamo construito il FF SR utilizzando delle porte NAND. Il circuito è riportato in Figura \ref e, come abbiamo già fatto per gli altri circuiti logici, il funzionamento è stato verificato con la basetta a LED. Come già visto a lezione, la tabella di verità è la seguente. 

\begin{figure}[H]
		\centering
		{\renewcommand{\arraystretch}{1.1}%
		\begin{tabular}{|c|c|c|c|c|c|}
		\hline
		S & R & $\bar S$ & $\bar R$ & $Q_{n+1}$ & $\bar Q_{n+1}$ \\
		\hline \hline
		0 & 0 & 1 & 1 & $Q_n$ & $\bar Q_n$\\
		\hline
		0 & 1 & 1 & 0 & 0 &1\\
		\hline
		1 & 0 & 0 & 1& 1 & 0\\
		\hline
		1 & 1 &0 &0 & ? & ?\\
		\hline
		\end{tabular}}
		\label{tab11:FFSR}
		\caption{puzzi}
        \end{figure}


Abbiamo collegato entrambe le uscite del Flip-Flop alla basetta al led e abbiamo notato che non avevamo mai due led accessi contemporaneamente, eccetto nel caso in cui sia Set che Reset erano ad 1 logico. In questo caso entrambi i led erano accessi. Ciò ovviamente non va bene per i nostri scopi, in quanto vogliamo avere uno stato stabile e un'uscita la negazione dell'altra. 

\subsubsection*{FF SR con Clock}

In questo secondo circuito vogliamo rendere sincrono quello precedente, inserendo un attivatore (CLK in  Tabella \ref{tab11:FFSR}). Portando come segnale dell'attivatore una forma d'onda quadra potremmo anche creare un clock. La tabella di verità è quella sottostante

\begin{figure}[H]
		\centering
		{\renewcommand{\arraystretch}{1.1}%
		\begin{tabular}{|c||c|c|c|c|}
		\hline
		CLK & S & R & $Q_{n+1}$ & $\bar Q_{n+1}$  \\
		\hline \hline
		0 & X & X & $Q_n$ & $\bar Q_n$\\
		\hline \hline
		 1&0 & 0 & $Q_n$ & $\bar Q_n$\\
		\hline
		1&0 & 1 & 0 &1\\
		\hline
		1&1 & 0 & 1 & 0\\
		\hline
		1&1 & 1 & ? & ?\\
		\hline
		\end{tabular}}
		\label{tab11:FFSR}
		\caption{puzzi di più}
        \end{figure}

Notiamo che, anche in questo caso, se sia S che R sono ad 1 logico, $Q$ e $\bar Q$ sono indeterminati.

\subsubsection*{Latch tipo D}


\begin{figure}[H]
		\centering
		{\renewcommand{\arraystretch}{1.1}%
		\begin{tabular}{c|c|c|c}
		D & CLK & $Q_{n+1}$ & $\bar Q_{n+1}$  \\
		\hline
		X & 0  & $Q_n$ & $\bar Q_n$\\
		\hline
		 1&1 & 1 & 0\\
		\hline
		0&1 & 0  &1\\
		\end{tabular}}
		\label{tab11:Latch_D}
		\caption{puzzi ancora di più}
        \end{figure}

Abbiamo inoltre notato che, con CLK ad 1, lasciando l'ingresso D flottante l'uscita Q era bassa. Ciò è in contraddizione con le porte integrate, in quanto quando un ingresso non è collegato a nessun riferimento viene visto come alto. Abbiamo capito che ciò era dovuto al fatto che abbiamo collegato il cavo dati alla basetta al led per vedere quando esso era a 0 o 1 logico. La basetta al led è costruita con resistenze di pull-down così che quando il segnale in ingresso è basso i led sono spenti. Dunque se monitoriamo il segnale D con la basetta, esso è collegato con resistenze di pull-down a comune. Abbiamo infatti verificato che scollegandolo dalla basetta, l'ingresso D flottante causava un uscita Q alta.

\subsubsection*{Circuito anti-rimbalzo}


\begin{figure}[H]
		\centering
		{\renewcommand{\arraystretch}{1.1}%
		\begin{tabular}{c|c|c|c}
		$\bar S$ & $\bar R$ & $Q_{n+1}$ & $\bar Q_{n+1}$  \\
		\hline
		0 & 0  & ?&?\\
		\hline
		0&1 & 1 & 0\\
		\hline
		1&0 & 0  &1\\
		\hline
		1&1 & $Q_n$ & $\bar Q_n$\\
		\end{tabular}}
		\label{tab11:antirimb}
		\caption{puzzi ancora di più di buzz}
        \end{figure}

\subsection*{Comandi separati Marcia/Arresto}




\subsubsection*{FF SR con porte NOR}


\begin{figure}[H]
		\centering
		{\renewcommand{\arraystretch}{1.1}%
		\begin{tabular}{c|c|c|c}
		S & R & $Q_{n+1}$ & $\bar Q_{n+1}$  \\
		\hline
		1 & 1  & ?&?\\
		\hline
		1&0 & 1 & 0\\
		\hline
		0&1 & 0  &1\\
		\hline
		0&0 & $Q_n$ & $\bar Q_n$\\
		\end{tabular}}
		\label{tab11:nor}
		\caption{puzzi come miani}
        \end{figure}


\subsubsection*{Divisore di frequenze}




