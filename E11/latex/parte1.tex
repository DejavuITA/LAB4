\subsection{Flip-Flop}

\subsubsection*{FF SR con porte NAND}

In questa prima parte dell'esperienza abbiamo construito il FF SR utilizzando delle porte NAND. Il circuito è riportato in Fig. e, come abbiamo già fatto per gli altri circuiti logici, il funzionamento è stato verificato con la basetta a LED. Come già visto a lezione, la tabella di verità è la seguente. 

\begin{figure}[H]
		\centering
		{\renewcommand{\arraystretch}{1.1}%
		\begin{tabular}{c|c|c|c}
		
		S & R & Q & $\bar Q$ \\
		\hline
		0 & 0 & Q & $\bar Q$\\
		\hline
		0 & 1 & 0 &1\\
		\hline
		1 & 0 & 1 & 0\\
		\hline
		1 & 1 & ? & ?\\
		\end{tabular}}
		\label{tab11:FFSR}
		\caption{puzzi}
        \end{figure}


Abbiamo collegato entrambe le uscite del Flip-Flop alla basetta al led e abbiamo notato che non avevamo mai due led accessi contemporaneamente, eccetto nel caso in cui sia Set che Reset erano ad 1 logico. In questo caso entrambi i led erano accessi. Ciò ovviamente non va bene per i nostri scopi, in quanto vogliamo avere uno stato stabile e un'uscita la negazione dell'altra. 


\subsubsection*{FF SR con Cock}






\begin{figure}[H]
		\centering
		{\renewcommand{\arraystretch}{1.1}%
		\begin{tabular}{c|c|c|c|c}
		CLK & S & R & Q & $\bar Q$ \\
		\hline		
		0 & ? & ? & Q & $\bar Q$\\
		\hline
		 1&0 & 0 & Q & $\bar Q$\\
		\hline
		1&0 & 1 & 0 &1\\
		\hline
		1&1 & 0 & 1 & 0\\
		\hline
		1&1 & 1 & ? & ?\\
		\end{tabular}}
		\label{tab11:FFSR}
		\caption{puzzi di più}
        \end{figure}





