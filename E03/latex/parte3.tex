\subsection{Verifica della banda passante}

In questa parte dell'esperienza vogliamo valutare la banda passante di un circuito amplificatore (non invertente) considerando che il guadagno a maglia aperta dell'amplificatore operazionale dipende della frequenza.

Esaminiamo dapprima il circuito, calcolandone la funzione di trasferimento. Per ogni amplificatore operazionale vale
\begin{wrapfigure}[20]{l}{0.55\textwidth}
  \begin{center}
    \includegraphics[width=0.30\textwidth]{../E03/latex/bandwidth.pdf}
  \end{center}
  \caption{Schema del circuito utilizzato per valutare la banda passante. La resistenza $R_1=997\pm1$\si{\ohm} è fissa; mentre $R_2$ è stata cambiata con valori $R_2^{11x}=9.91 \pm 0.01$ \si{\kilo\ohm} e $R_2^{11x}=99.1 \pm 0.1$ \si{\kilo\ohm} a seconda del guadagno desiderato.}
  \label{cir3:banda}
\end{wrapfigure}
\begin{equation}
V_{out}=A(s) (V^+-V^-)
\label{eq3:regola_opamp}
\end{equation}
con il guadagno, come detto sopra, che in generale dipende da $s=j\omega$ (e quindi dalla frequenza). Consideriamo
$$V^+ = V_{in} \qquad V^-=V_{out} \frac{R_1}{R_1+R_2}$$
dove $V^-$ è ricavato dalla solita formula per l'amplificatore non invertente $(V^+-V^-)/R_1 + (V_{out}-V^-)/R_2 =0$. Definiamo
$$\beta = \frac{R_1}{R_1+R_2} = \frac{1}{G}$$
dove $G$ è il guadagno di un amplificatore non invertente. Sostituendo questi valori in (\ref{eq3:regola_opamp}) otteniamo
$$A(s) V_{in} = V_{out} + V_{out} A(s) \beta$$
Calcoliamo ora la funzione di trasferimento $H$
\begin{equation}
H(s)=\frac{V_{out}}{V_{in}}=\frac{1}{\beta}\frac{1}{1+\frac{1}{A(s) \beta}}
\label{eq3:funz_trasfe}
\end{equation}
da cui è facile notare che per $A(s) \rightarrow + \infty$ (approssimazione di amplificatore ideale), $H(s)=\frac{1}{\beta}=1+\frac{R_2}{R_1}=G$, equazione che diventa indipendente dal guadagno a maglia aperta.

Schematizziamo l'OPAMP come un filtro passa basso. Abbiamo che il guadagno a maglia aperta varia con la frequenza secondo la legge
$$A(j\omega)=\frac{A_{ol}}{1+j\frac{\omega}{\omega_0}}$$
con $\omega_0 \approx 8$ \si{\hertz} la prima frequenza di taglio dell'operazionale data dalla capacità di compensazione nella circuiteria interna e $A_{ol}$ il guadagno dell'operazionale con segnali costanti ($f = 0$). Sostituendo questo valore in (\ref{eq3:funz_trasfe}), abbiamo
$$H(s)=\frac{\frac{A_{ol}}{1+A_{ol}\beta}}{1+j \frac{\omega}{(1+A_{ol}\beta)\omega_0}}$$
da cui si nota che la nuova frequenza di taglio è data da $\omega_f=(1+A_{ol}\beta)\omega_0$. Nel circuito in Figura \ref{cir3:banda} avevamo un guadagno $G$ retroazionato di $11$ e $101$, quindi ci aspettiamo\footnote{Questi valori sono approssimati perchè cercheremo di calcolare $A_{ol}$ con le frequenze a nostra disposizione, ricavate con l'analisi della banda, piuttosto che calcolare ora frequenze di taglio senza essere certi del valore di $A_{ol}$.} frequenze di taglio, per l'equazione precedente, rispettivamente sull'ordine dei $10^5$ e $10^4$.

