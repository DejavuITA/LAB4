\section{30.09.2014 - Amplificatori Operazionali Reali - Seconda Parte}

In questa esperienza abbiamo studiato ulteriori caratteristiche che distinguono l'opamp reale dal modello ideale.
Più precisamente abbiamo analizzato e misurato lo \textit{slew rate}, la massima corrente pilotabile, la larghezza di banda passante e il guadagno a maglia aperta $A_{ol}$ di un opamp $\mu$A741.

Ricordiamo che in ogni schema sono stati sottintesi la circuitazione di alimentazione (cavi e capacitori) e quella dedicata al bilanciamento dell'offset (cavi e trimmer).

\subsection{Strumenti e materiali}

\begin{itemize} [noitemsep]
\item Oscilloscopio Agilent DSO-X 2002A (bandwidth \SI{70}{\mega\hertz}, sample rate \num{2} GSa/s);
\item Generatore di tensione continua Agilent E3631A (max $\pm \, \SI{25}{\volt}$ o $\pm \, \SI{6}{\volt}$);
\item Generatore di forme d'onta Agilent 33120A con range di frequenza da \SI{100}{\micro\hertz} a \SI{15}{\mega\hertz};
\item Multimetro Agilent 34410A;
\item Un amplificatore operazionale $\mu$A741;
\item Resistenze e capacità di vari valori;
%\item un trimmer (potenziometro);
\item Breadboard e cablaggi vari.
\end{itemize}
